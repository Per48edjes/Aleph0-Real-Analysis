\section{Sequences and Limits}

\begin{problem}
  Verify, using the definition of convergence of a sequence, that the following
  sequences converge to the proposed limit.

  \begin{enumerate}[label=(\alph*)]
    \item $\lim\limits_{n \to \infty} \dfrac{1}{6n^{2} + 1} = 0$

      Let $\eps > 0$ be given. We need to find $N \in \N$ such that for all $n > N$, we have 
      \[
        \abs{\frac{1}{6n^{2} + 1}} < \eps.
      \]
      Choose $N > \sqrt{\frac{1}{6}(\frac{1}{\eps} - 1)}$. Then for all $n > N$,
      \[
        \abs{\frac{1}{6n^{2} + 1}} = \frac{1}{6n^{2} + 1} < \frac{1}{6 \left( \sqrt{\frac{1}{6}(\frac{1}{\eps} - 1)} \right)^{2} + 1} = \eps.
      \]

    \item $\lim\limits_{n \to \infty} \dfrac{3n + 1}{2n + 5} = \dfrac{3}{2}$

      Let $\eps > 0$ be given. We need to find $N \in \N$ such that for all $n > N$, we have 
      \[
        \abs{\frac{3n + 1}{2n + 5} - \frac{3}{2}} < \eps.
      \]
      Simplifying the expression, we have
      \[
        \abs{\frac{3n + 1}{2n + 5} - \frac{3}{2}} = \abs{\frac{6n + 2 - 6n - 15}{2(2n + 5)}} = \abs{\frac{-13}{2(2n + 5)}} = \frac{13}{2(2n + 5)},
      \]
      so having $N$ large enough such that
      \[
        \frac{13}{2(2N + 5)} < \eps
      \]
      is sufficient. Rearranging the inequality gives 
      \[
        2(2N + 5) > \frac{13}{\eps} \implies N > \frac{13}{4\eps} - \frac{5}{2}.
      \]
      Thus, we can choose $N > \max\left(0, \frac{13}{4\eps} - \frac{5}{2}\right)$ to 
      guarantee that for all $n > N$,
      \[
        \abs{\frac{3n + 1}{2n + 5} - \frac{3}{2}} < \eps.
      \]
  \end{enumerate}

\end{problem}

\begin{problem}
  Let $(a_{n})$ be a sequence. Prove that if $\lim\limits_{n \to \infty} \abs{a_{n}} = 0$, 
  then $\lim\limits_{n \to \infty} a_{n} = 0$.

  \begin{proof}
    Since $\lim\limits_{n \to \infty} \abs{a_{n}} = 0$, for every $\eps > 0$, there exists $N$ 
    such that for all $n > N$, we have
    \[
      \abs{ \abs{ a_{n} } } = \abs{a_{n}} < \eps,
    \]
    which is precisely the definition of convergence of the sequence $(a_{n})$ to $0$, i.e., 
    $\lim\limits_{n \to \infty} a_{n} = 0$.
  \end{proof}

\end{problem}

\begin{problem}
  Let $(a_{n})$ be a sequence. Prove that if $\lim\limits_{n \to \infty} a_{n} = L$ 
  and $\lim\limits_{n \to \infty} a_{n+1} = M$, then $L = M$.

  \begin{proof}
    Suppose to the contrary that $L \neq M$. Without loss of generality, assume $L < M$. Then
    we can choose $\eps = \frac{M - L}{2} > 0$. By the definition of convergence, there exists $N_{1}$ such that for all $n > N_{1}$,
    \[
      \abs{a_{n} - L} < \frac{M - L}{2},
    \]
    and there exists $N_{2}$ such that for all $n > N_{2}$,
    \[
      \abs{a_{n+1} - M} < \frac{M - L}{2}.
    \]
    Let $N = \max(N_{1}, N_{2})$. For all $n > N$, we have
    \begin{equation}
      a_{n} < L + \frac{M - L}{2} = \frac{L + M}{2} \quad \text{and} \quad a_{n+1} > M - \frac{M - L}{2} = \frac{L + M}{2}. \label{eqn:contradiction-limits-are-unique}
    \end{equation}
    Since $\{ a_{n} : n > N \}$ and $\{ a_{n+1} : n > N \}$ are tails of the same sequence, they must share some common element,
    but \eqref{eqn:contradiction-limits-are-unique} implies that $\{ a_{n} : n > N \} \cap \{ a_{n+1} : n > N \} = \emptyset$.

    Having reached a contradiction, our assumption that $L \neq M$ must be false, and we conclude that $L = M$.
  \end{proof}

\end{problem}

\begin{problem}
  Let $(a_{n})$ be a sequence and $L$ be a real number. Prove that
  $\lim\limits_{n \to \infty} a_{n} = L$ iff $\lim\limits_{n \to \infty} \abs{a_{n} - L} = 0$.

  \begin{proof}
    Recall that $\lim\limits_{n \to \infty} a_n = L$ means that for all $\eps > 0$, there exists $N \in \N$ such that for all $n > N$,
    \[
      \abs{a_{n} - L} < \eps.
    \]
    But this is exactly the definition of $\lim\limits_{n \to \infty} |a_n - L| = 0$, since
    \[
      \abs{\abs{a_{n} - L} - 0} = \abs{a_{n} - L}.
    \]
    Thus, the two statements are equivalent.
  \end{proof}
\end{problem}

\begin{problem}
  If $(a_{n})$ is a convergent sequence, with $\lim\limits_{n \to \infty} a_{n} = L$, prove that
  $\abs{a_{n}}$ converges as well. Does $\abs{a_{n}}$ necessarily converge to $\abs{L}$?

  \begin{proof}
    Since $\lim\limits_{n \to \infty} a_n = L$, for every $\eps > 0$, there exists $N \in \N$ such that for all $n > N$,
    \[
      \abs{a_{n} - L} < \eps.
    \]
    By the reverse triangle inequality, we have
    \[
    \abs{\abs{a_{n}} - \abs{L}} \leq \abs{a_{n} - L} < \eps.
    \]
    Therefore, $\lim\limits_{n \to \infty} \abs{a_{n}} = \abs{L}$, and so $(\abs{a_{n}})$ converges to $\abs{L}$\footnotemark.
  \end{proof}
  \footnotetext{
    This result is consistent with the fact that $f(x) = \abs{x}$ is continuous everywhere in $\R$ and
    by continuity, $\lim\limits_{n \to \infty} \abs{a_{n}} = \lim\limits_{n \to \infty} f(a_n) = f(\lim\limits_{n \to \infty} a_n) = \abs{L}$.
  }
\end{problem}

\begin{problem}
  Verify, using the definition of convergence, that the following sequences converge to the proposed limit.

  \begin{enumerate}[label=(\alph*)]
    \item $\lim\limits_{n \to \infty} \dfrac{n^{2} + 1}{2n^{2} + 3} = \dfrac{1}{2}$

      Let $\eps > 0$ be given. We need to find $N \in \N$ such that for all $n > N$, we have 
      \[
        \abs{\frac{n^{2} + 1}{2n^{2} + 3} - \frac{1}{2}} < \eps.
      \]
      Simplifying the expression, we have
      \[
        \abs{\frac{n^{2} + 1}{2n^{2} + 3} - \frac{1}{2}} = \abs{\frac{2n^{2} + 2 - 2n^{2} - 3}{2(2n^{2} + 3)}} = \abs{\frac{-1}{4n^{2} + 6}} = \frac{1}{4n^{2} + 6},
      \]
      so having $N$ large enough such that
      \[
        \frac{1}{4N^{2} + 6} < \eps
      \]
      is sufficient. Rearranging the inequality gives 
      \[
        N > \sqrt{\frac{1}{4}(\frac{1}{\eps} - 6)}.
      \]
      Thus, we can choose $N > \sqrt{\frac{1}{4}(\frac{1}{\eps} - 6)}$ to 
      guarantee that for all $n > N$,
      \[
        \abs{\frac{n^{2} + 1}{2n^{2} + 3} - \frac{1}{2}} < \eps.
      \]

    \item $\lim\limits_{n \to \infty} \dfrac{3^{n}}{3^{n} + 2^{n}} = 1$

      Let $\eps > 0$ be given. We need to find $N \in \N$ such that for all $n > N$, we have 
      \[
        \abs{\frac{3^{n}}{3^{n} + 2^{n}} - 1} < \eps.
      \]
      Simplifying the expression, we have
      \[
        \abs{\frac{3^{n}}{3^{n} + 2^{n}} - 1} = \abs{\frac{1 - (1 + \left( \frac{2}{3} \right)^{n})}{1 + \left( \frac{2}{3} \right)^{n}}} = \frac{\left( \frac{2}{3} \right)^{n}}{1 + \left( \frac{2}{3} \right)^{n}} < \eps.
      \]
      Notice that if $\eps \geq 1$, $N = 1$ is sufficient, so we restrict our attention to when $\eps < 1$.
      
      If we can find some $N$ satisfying
      \[
        \left(\frac{2}{3}\right)^{N} < \eps,
      \]
      we are done because for all $N \in \N$,
      \[
        \frac{\left( \frac{2}{3} \right)^{N}}{1 + \left( \frac{2}{3} \right)^{N}} < \left(\frac{2}{3}\right)^{N}.
      \]
      Taking the natural logarithm of both sides, we obtain
      \[ 
        N \ln{ \left( \frac{2}{3} \right) } < \ln{\eps} \implies N > \frac{\ln (\eps)}{\ln \left( \frac{2}{3} \right)}.
      \]
      Thus, we can choose $N > \frac{\ln (\eps)}{\ln \left( \frac{2}{3} \right)}$ to guarantee that for all $n > N$,
      \[
        \abs{\frac{3^{n}}{3^{n} + 2^{n}} - 1} < \eps.
      \]

    \item $\lim\limits_{n \to \infty} \dfrac{\sin n}{n} = 0$

      Let $\eps > 0$ be given. We need to find $N \in \N$ such that for all $n > N$, we have 
      \[
        \abs{\frac{\sin n}{n}} < \eps.
      \]
      If we can find some $N$ satisfying
      \[
        \frac{1}{N} < \eps,
      \]
      we are done because for all $N \in \N$,
      \[
        \abs{\frac{\sin N}{N}} < \dfrac{1}{N}.
      \]
      Choosing $N > \dfrac{1}{\eps}$ guarantees that for all $n > N$,
      \[
        \abs{\frac{\sin n}{n}} < \eps.
      \]

  \end{enumerate}

\end{problem}

\begin{problem}
  Consider the sequence $(a_{n})$ defined by $a_{n} = \dfrac{(-1)^{n} n}{n + 1}$.
  Determine whether this sequence converges or diverges. If it converges, find
  its limit.

  \begin{proof}
    We will show that $(a_{n})$ diverges. To begin, we will establish the following lemma:
    
    \begin{lemma} \label{lem:convergence-of-subsequences-iff-convergence-of-sequence}
      Sequence $(a_{n})$ converges to $L$ if and only if all subsequences of $(a_{n})$ converge to $L$.
    \end{lemma}

    \begin{subproof}
      We will begin by proving the left-to-right implication. Let $(a_{n}) \to L$ 
      and $(a_{n_{k}})$ be an arbitrary subsequence of $(a_{n})$.
      Let $\eps > 0$ be given. We want to show that 
      \[
        \forall \eps > 0.\, \exists K \in \N.\,  k > K \implies \abs{a_{n_{k}} - L} < \eps.
      \]
      By convergence, there exists some $N \in \N$ such that for all $n > N$ we have
      \[
        \abs{a_{n} - L} < \eps.
      \]
      Now, for any point in $(a_{n_{k}})$, it can be shown (via induction) that 
      its index $k$ (in the subsequence) is always less than or equal to its
      corresponding index $n_{k}$ in the parent sequence $(a_{n})$. Let $K = N$. Then, 
      $k > K = N \implies n_{k} \geq k > N$, so we have found a $K \in \N$ satisfying 
      \[
        \abs{a_{n_{k}} - L} < \eps
      \]
      whenever $k > K$.

      Finally, the right-to-left implication is proven by using the fact that
      $(a_{n})$ is a subsequence of itself and the hypothesis that all
      subsequences of $(a_{n})$ converge to $L$. Hence, $a_{n}$ converges to $L$.
    \end{subproof}

    Consider the following subsequences:
    \begin{enumerate}[label=(\roman*)]
      \item $(a_{2k - 1}) = (-\frac{1}{2}, -\frac{3}{4}, -\frac{5}{6}, \ldots)$, i.e., the odd-indexed terms of the $(a_{n})$
      \item $(a_{2k}) = (\frac{2}{3}, \frac{4}{5}, \frac{6}{7} \ldots)$, i.e., the even-indexed terms of the $(a_{n})$
    \end{enumerate}
    By the definition of convergence, it can be shown that $(a_{2k - 1}) \to -1$ and $(a_{2k}) \to 1$.
    Hence, by \lemref{lem:convergence-of-subsequences-iff-convergence-of-sequence}, $(a_{n})$ does not converge.
  \end{proof}


\end{problem}
