\section{Upper and Lower Bounds}

\begin{problem}\label{prob:bounded-convergent}
  Prove that if $a$ is an upper bound for $A$, and if $a$ is also an element of $A$, then it must be that $a = \sup{A}$.
\end{problem}

\begin{proof}
  To establish that $a = \sup{A}$, we must show that:
  \begin{enumerate}
    \item $a$ is an upper bound for $A$, and \label{item:upper-bound}
    \item $a \leq b$ for any upper bound $b$ of $A$. \label{item:least-upper-bound}
  \end{enumerate}

  Note that (\ref{item:upper-bound}) is given by the problem statement, so we
  only need to show (\ref{item:least-upper-bound}). Immediately, $a$ is the
  least upper bound of $A$ because any $b > a$ (whilst being an upper bound of
  $A$) would not be the \textit{least} upper bound (since $a$ is a smaller upper
  bound), and any $b < a$ cannot be a smaller upper bound of $A$ (since $a$ is an
  element of $A$ and thus $b$ would not be greater than or equal to all
  elements of $A$).
\end{proof}


\begin{problem}\label{prob:sup-subset}
  Let $A$ and $B$ be non-empty sets. Prove that if $B \subseteq A$, then $\sup{B} \leq \sup{A}$.
\end{problem}

\begin{proof}
  Let $a = \sup{A}$ and $b = \sup{B}$. We need to show that $b \leq a$.

  Since $B \subseteq A$, every element of $B$ is also an element of $A$. Thus,
  by the definition of supremum, $b$ must be less than or equal to any upper
  bound of $A$. Since $a$ is an upper bound for $A$, we have:
  \[
    b \leq a.
  \]
  Therefore, we conclude that $\sup{B} \leq \sup{A}$.
\end{proof}
