\section{Upper and Lower Bounds}

\begin{problem}\label{prob:bounded-convergent}
  Prove that if $a$ is an upper bound for $A$, and if $a$ is also an element of $A$, 
  then it must be that $a = \sup{A}$.
\end{problem}

\begin{proof}
  To establish that $a = \sup{A}$, we must show that:
  \begin{enumerate}
    \item $a$ is an upper bound for $A$, and \label{item:upper-bound}
    \item $a \leq b$ for any upper bound $b$ of $A$. \label{item:least-upper-bound}
  \end{enumerate}

  Note that (\ref{item:upper-bound}) is given by the problem statement, so we
  only need to show (\ref{item:least-upper-bound}). Immediately, $a$ is the
  least upper bound of $A$ because any $b > a$ (whilst being an upper bound of
  $A$) would not be the \textit{least} upper bound (since $a$ is a smaller upper
  bound), and any $b < a$ cannot be a smaller upper bound of $A$ (since $a$ is an
  element of $A$ and thus $b$ would not be greater than or equal to all
  elements of $A$).
\end{proof}


\begin{problem}\label{prob:sup-subset}
  Let $A$ and $B$ be nonempty sets. Prove that if $B \subseteq A$, then
  $\sup{B} \leq \sup{A}$.
\end{problem}

\begin{proof}
  Let $a = \sup{A}$ and $b = \sup{B}$. We need to show that $b \leq a$.

  Since $B \subseteq A$, every element of $B$ is also an element of $A$. Thus,
  by the definition of supremum, $b$ must be less than or equal to any upper
  bound of $A$. Since $a$ is an upper bound for $A$, we have:
  \[
    b \leq a.
  \]
  Therefore, we conclude that $\sup{B} \leq \sup{A}$.
\end{proof}


\begin{problem}\label{prob:sup-inf-practice}
  Compute, without proof, the supremum and infimum of the following sets:
  \begin{enumerate}[label=(\alph*)]
    \item $\{n \in \N : n^2 < 10\}$
      
    $\sup = 3, \quad \inf = 1$

    \item $\{n/(m + n) : m, n \in \N\}$

    $\sup = 1, \quad \inf = 0$

    \item $\{n/(2n + 1) : n \in \N\}$

    $\sup = 1/2, \quad \inf = 1/3$

    \item $\{n/m : m, n \in \N \text{ with } m + n \leq 10\}$

    $\sup = 9, \quad \inf = 1/9$
  \end{enumerate}
\end{problem}

\begin{problem}\label{prob:sup-inf-linear-transformation}
  Let $A \subseteq \R$ be a nonempty bounded set, and let $c \in \R$. Define the sets $c + A$ and $cA$ by 
  $c + A = \{c + a : a \in A\}$ and $cA = \{ca : a \in A\}$.
  \begin{enumerate}[label=(\alph*)]
    \item Show that $\sup(c + A) = c + \sup A$. \label{prob:sup-c+A}

      \begin{proof}
        Let $s = \sup A$. Since $A$ is nonempty and bounded above, $s$ exists. First we will show that $c + s$ is an upper bound for $c + A$, 
        and then we will show that it is the least upper bound.

        For every $a \in A$, we have $a \leq s$. Therefore, adding $c$ to both sides gives us:
        \[
          c + a \leq c + s,
        \]
        so every element of $c + A$ is less than or equal to $c + s$. This shows that $c + s$ is an upper bound for $c + A$.

        Next, suppose for sake of contradiction that $c + s$ is not the \textit{least} upper bound for $c + A$.
        This would mean $b = \sup(c + A) < c + s$ for some $b \in \R$. By the definition of the supremum, there exists some $a \in A$ such that:
        \[
          b - c < a,
        \] 
        and adding $c$ to both sides gives us:
        \[
          b < a + c,
        \]
        resulting in a contradiction since $c + a \in c + A$ and $b$ is supposed to be an upper bound for $c + A$.
        Therefore, no such $b$ can exist, and we conclude that $c + s$ is indeed the least upper bound for $c + A$.
      \end{proof}

    \item If $c \geq 0$, show that $\sup(cA) = c \sup A$. \label{prob:sup-cA}

      \begin{proof}
        Let $s = \sup A$. Since $A$ is nonempty and bounded above, $s$ exists. We will show that $c s$ is an upper bound for $cA$, and then 
        we will show that it is the least upper bound.

        For every $a \in A$, we have $a \leq s$. Therefore, multiplying both sides by $c \geq 0$ gives us:
        \[
          ca \leq cs,
        \]
        so every element of $cA$ is less than or equal to $cs$. This shows that $cs$ is an upper bound for $cA$. 
        Next, we proceed by casework on the value of $c$ to show that $cs$ is the least upper bound.

        When $c > 0$, let us suppose for sake of contradiction that $cs$ is not the \textit{least} upper bound for $cA$. 
        This means $b = \sup(cA) < cs$ for some $b \in \R$. First, consider when $c > 0$. Dividing through by $c > 0$ gives us:
        \[
          \frac{b}{c} < s,
        \]
        so there exists some $a \in A$ such that:
        \[
          a > \frac{b}{c} \implies ca > b.
        \]
        However, a contradiction is reached because $ca \in cA$ and $b$ is supposed to be an upper bound for $cA$.

        Now consider the case when $c = 0$. In this case, $cA = \{0\}$, and thus $\sup(cA) = 0 = cs$. 

        Therefore, in both cases, we conclude that $cs$ is indeed the least upper bound for $cA$.
      \end{proof}

    \item If $c < 0$, show that $\sup(cA) = c \inf A$. \label{prob:sup-cA-negative}

      \begin{proof}

        Let $c = -d$ where $d > 0$. Then we can rewrite the set $cA$ as $-dA = \{-da : a \in A\} = -(dA)$. 
        The following lemma will be useful:

        \begin{lemma}\label{lem:sup-neg-bounded}
          If $B \subseteq \R$ is nonempty and bounded below, then $\sup(-B) = -\inf B$ 
          where $-B = \{-b : b \in B\}$.
        \end{lemma}

        \begin{subproof}[Proof of \lemref{lem:sup-neg-bounded}]
          Let $m = \inf B$, so for every $b \in B$, we have $b \geq m$.
          Therefore, multiplying both sides by $-1$ (which reverses the
          inequality) gives us:
          \[
            -b \leq -m,
          \]
          so every element of $-B$ is less than or equal to $-m$, thus $-m$ is
          an upper bound for $-B$. Now, suppose $u < -m$ is an upper bound for
          $-B$. Then,  $-u > m$, which means there exists some $b \in B$ with
          $m \leq b < -u \implies -b > u$ (by the definition of the infimum),
          so $u$ cannot be an upper bound for $-B$.

          Hence, $-m = -\inf B$ is the least upper bound for $-B$. 
        \end{subproof}

        Invoking \lemref{lem:sup-neg-bounded} and the result from \ref{prob:sup-cA},
        we have $\sup(-dA) = -d \inf(A)$, and thus, $\sup(cA) = c \inf(A)$ when
        $c < 0$. 

      \end{proof}

  \end{enumerate}
\end{problem}

\begin{problem} \label{prob:sup-upper-bound}
  If $\sup A < \sup B$, then show that there exists some $b \in B$ that is an upper bound for $A$.
\end{problem}

\begin{proof}
  Let $s_A = \sup A$ and $s_B = \sup B$. Since $s_A < s_B$, there exists some
  $\eps$ such that $s_B - s_A > \eps > 0$. 

  Notice that there exists some $b \in B$ such that $s_B - \eps < b \leq s_B$
  (by the definition of the supremum) and $s_B - \eps > s_A$, so taken together
  we have 
  \[
    a \leq s_A < s_B - \eps < b,
  \]
  and therefore, some $b \in B$ is an upper bound for $A$.
\end{proof}

\begin{problem} \label{prob:sup-inf-claims}
  Without worrying about formal proofs for the moment, decide if the following
  statements about suprema and infima are true or false. For any that are false,
  supply an example where the claim does not appear to hold.

  \begin{enumerate}[label=(\alph*)]
    \item A finite, nonempty set always contains its supremum. 

      True.

    \item If $a < L$ for every element $a \in A$, then $\sup a < L$.

      False. It could be the case that $\sup A = L$. For example, let
      \[
        A = \{ a : a < 1 \}.
      \]

    \item If $A$ and $B$ are sets with the property that $a < b$ for every $a
      \in A$ and $b \in B$, then it follows that $\sup A < \sup B$.

      True.

    \item If $\sup A = s$ and $\sup B = t$, then $\sup (A + B) = \sup A + \sup B$.
      The set $A + B$ is defined as $A + B = \{a + b : a \in A \text{ and } b \in B\}$.

      True. 

    \item If $\sup A \leq \sup B$, then there exists an element of $b \in B$
      that is an upper bound for $A$. 

      False. Consider $A = B = \{ x : 0 < x < 1 \}$. Then $\sup A = \sup B = 1$,
      but no element $b \in B$ is an upper bound for $A$ because no element
      in $b$ is greater than or equal to $1$.

  \end{enumerate}

\end{problem}

\begin{problem}\label{prob:sup-A-minus-B}
  Let $A$ and $B$ be nonempty subsets of $\R$ with $A$ bounded above and $B$
  bounded below. Define the set $A - B = \{ a - b : a \in A, b \in B\}$.
  Prove that $\sup(A-B) = \sup A - \inf B$.
\end{problem}

\begin{proof}
  Let $s = \sup A - \inf B$. First, we see that $s$ is an upper bound
  for $A - B$ since there does not exist any $x \in A - B$ such that $x > s$.

  Now we must show that $s$ is the \textit{least} upper bound for $A - B$. Suppose,
  to the contrary, that $\exists u \in \R$ such that $x \leq u < s$ for any $x \in A - B$.
  Then, $\exists \eps > 0$ satisfying $s - \eps = u$, giving:
  \begin{align*}
    u = s - \eps &= \sup A - \inf B - \eps \\
                 &= (\sup A - \frac{\eps}{2}) - (\inf B + \frac{\eps}{2}). \\ 
  \end{align*}
  Notice that, by definition of the supremum and infimum, respectively,
  $\exists a' \in A$ such that $a' > (\sup A - \frac{\eps}{2})$ and $\exists b'
  \in B$ such that $b' < (\inf B + \frac{e}{2})$, which results in a
  contradiction: $u < a' - b' \in A - B$. Therefore, $s$ must be the least
  upper bound for $A - B$.
\end{proof}

\begin{problem}
  Prove that for any nonempty set $S$ of real numbers that is bounded below,
  $\inf S = - \sup(-S)$, where $-S = \{ -s : s \in S\}$.
\end{problem}

\begin{proof}
  See proof of \lemref{lem:sup-neg-bounded} in \ref{prob:sup-inf-linear-transformation}\ref{prob:sup-cA-negative}.
\end{proof}

\begin{problem}
  For a bounded set $A \subset \R$, define its diameter as $\diam{A} = \sup A -
  \inf A$. Prove that for any bounded sets $A$ and $B$, $\diam{A \cup B} \leq
  \diam{A} + \diam{B}$ if and only if there exists $c \in A \cap B$.
\end{problem}

\begin{callout}
  This claim is incorrect. Consider $A = \{2, 4, 6, 8\}$ and $B = \{1, 3, 5, 7, 9, 11 \}$.
  Then, $\diam{A \cup B} \leq \diam{A} + \diam{B}$ even when $\nexists c \in A \cap B$. 
\end{callout}

\begin{problem}\label{prob:inf-of-upper-bounds-is-sup}
  Let $A$ be a nonempty subset of $\R$ that is bounded above. Define 
  $S = \{ x : x \text{ is an upper bound for } A \}$. Prove that $\inf S = \sup A$.
\end{problem}

\begin{proof}
  We argue by contradiction that $\inf S = \sup A$:
  \begin{itemize}
    \item If we let $\eps = \sup A - \inf S > 0$ and $c = \inf S +
      \frac{\eps}{2} = \sup A - \frac{\eps}{2}$, then there exists some
      element $a' \in A$ where $a' > c$ and some element $x' \in S$ such that $x' < c$, which
      results in $a' > x'$, contradicting the fact that $S = \{ x : \forall a \in A.\ x \geq a \}$ 
      only contains upper bounds of $A$.
    \item If we let $\inf S > \sup A$, then there would exist some point $c$ such that $\inf S > c > \sup A$, 
      implying $c$ is an upper bound of $A$ but not in $S$ -- a contradiction.
  \end{itemize}
  Hence, $\inf S = \sup A$.
\end{proof}

\begin{problem}\label{prob:density-rationals-in-reals}
  Show that between any two distinct real numbers, there is a rational number.
\end{problem}

\begin{proof}
  Without loss of generality, consider arbitrary $x_{1}, x_{2} \in \R$ with
  $x_{1} < x_{2}$. For convenience, we define $\Delta = x_{2} - x_{1}$. Notice
  that if $\Delta > 1$, $\exists r \in \Z$ such that $x_1 < r < x_2$. We must
  show that $\exists r \in \Q$ such that $x_{1} < r < x_{2}$ when $\Delta \leq 1$.

  Consider the case when $\Delta \leq 1$ and $x_{1} \in \Q$ or $x_{2} \in \Q$:
  \begin{itemize}
    \item If $x_{1} \in \Q$, and we let $S = \{ 1/n : n \in \N, n > 0 \}$ and
      $0 < \eps < \Delta$, then $\exists s \in S$ such that 
      $x_{1} < x_{1} + s < x_{1} + \inf S + \eps = x_{1} + \eps < x_{2}$ (by the definition 
      of the infimum). Notice that $x_1 + s \in \Q$, so we have some rational 
      $r = x_{1} + s$ such that $x_{1} < r < x_{2}$. 
    \item A similar argument can be made when $x_{2} \in \Q$. Let $-S = \{ -1/n : n \in \N, n > 0 \}$
      and $0 < \eps < \Delta$: $\exists s \in -S$ such that 
      $x_{1} < x_{2} + \sup(-S) - \eps = x_{2} - \eps < x_{2} + s < x_{2}$ 
      (by the definition of the supremum). So, $\Q \ni r = x_{2} + s$ satisfies $x_{1} < r < x_{2}$.
  \end{itemize}

  Finally, we have the case when $\Delta \leq 1$ and $x_{1}, x_{2} \not\in \Q$.
  In the subcase where $\lceil x_{2} \rceil - \lfloor x_{1} \rfloor > 1$,
  we have $x_{1} < r = \lceil x_{1} \rceil = \lfloor x_{2} \rfloor < x_{2}$ 
  (since $r \in \Z$), so the remainder of the proof concerns itself with the
  last subcase (i.e., when $\lceil x_{2} \rceil - \lfloor x_{1} \rfloor = 1$).

  Let $n \in \N$ so that $\frac{1}{n} < \Delta$ (such a $n$ exists by the Archimedean property). 
  Then, $\exists k \in \Z$ such that $x_{1} < \frac{k}{n} < x_{2}$ because $nx_{1} < k < nx_{2}$, 
  where $k$ is some integer (owing to the fact that $n \cdot \Delta > 1$). 
  Since $\frac{k}{n} \in \Q$, we have found some rational $r = \frac{k}{n}$ such that $x_{1} < r < x_{2}$. 
\end{proof}

\begin{problem}\label{prob:infinite-rationals-in-reals}
  Show that between any two distinct real numbers, there are infinitely many rational numbers.
\end{problem}

\begin{proof}
  Without loss of generality, let $x_{1}, x_{2} \in \R$ be arbitrary with $x_{1} < x_{2}$. Suppose, for 
  contradiction, that there are only \textit{finitely} many rational numbers between $x_{1}$ and $x_{2}$. Let
  \[
    S = \{r \in \Q : x_{1} < r < x_{2} \}.
  \]
  Since $S$ is nonempty and finite, there exists a maximum element $M \in S$
  such that $M \geq s$ for all $s \in S$. However, $M < x_{2}$ and $M, x_{2} \in \R$, 
  so by the claim made in \probref{prob:density-rationals-in-reals}, there exists $M' \in \Q$ 
  such that $M < M' < x_{2}$. This leads to a contradiction:
  \begin{itemize}
    \item $M$ is the maximum rational number between $x_{1}$ and $x_{2}$, but
    \item $M' > M$ and is a rational number between $x_{1}$ and $x_{2}$.
  \end{itemize}
  Therefore, our assumption that there are finitely many rational numbers
  between $x_{1}$ and $x_{2}$ is false. Hence, there are infinitely many rational
  numbers between any two distinct real numbers.
\end{proof}

\begin{problem}\label{prob:density-irrationals-in-reals}
  Prove that between any two distinct real numbers, there exists a irrational number.
\end{problem}

\begin{proof}
  We start by proving two, useful lemmas that will aid us later in the proof.

    \begin{lemma}\label{lem:rationals-closed-under-addition-multiplication}
      If $a, b \in \Q$, then $a + b \in \Q$ and $ab \in \Q$.
    \end{lemma}

      \begin{subproof}[Proof of \lemref{lem:rationals-closed-under-addition-multiplication}]
        Let $a, b \in \Q$ be arbitrary. We may express $a = \frac{p_{a}}{q_{a}}$ and $b = \frac{p_{b}}{q_{b}}$, 
        where $p_{a}, p_{b} \in \Z$ and $q_{a}, q_{b} \in \N$ such that $\gcd(p_{a}, q_{a}) = 1$ and $\gcd(p_{b}, q_{b}) = 1$.
        First, we can show that $ab \in \Q$:
        \[
          ab = \frac{p_{a}p_{b}}{q_{a}q_{b}},
        \]
        and $p_{a}p_{b} \in \Z$ and $q_{a}q_{b} \in \N$ with $\gcd(p_{a}p_{b}, q_{a}q_{b}) = 1$, so $ab \in \Q$.

        Secondly, we can show that $a + b \in \Q$:
        \[
          a + b = \frac{p_{a}}{q_{b}} + \frac{p_{b}}{q_{a}} = \frac{p_{a}q_{b} + p_{b}q_{a}}{q_{a}q_{b}}
        \]
        and $p_{a}q_{b} + p_{b}q_{a} \in \Z$ and $q_{a}q_{b} \in \N$ with 
        $\gcd(p_{a}q_{b} + p_{b}q_{a}, q_{a}q_{b}) = 1$, so $a + b \in \Q$.

        Therefore, the rationals are closed under addition and multiplication. 
      \end{subproof}

    \begin{lemma}\label{lem:rational-irrational-addition-multiplication}
      If $a \in \Q$ and $b \in \I$\footnotemark{}, then $a + b \in \I$ and $ab \in \I$ as long as $a \neq 0$.
    \end{lemma}
    \footnotetext{$\I$ is defined as $\R \setminus \Q$.}

    \begin{subproof}[Proof of \lemref{lem:rational-irrational-addition-multiplication}]
      Let $\Q \ni a \neq 0$ and $b \in \I$ be arbitrary. 
      First, suppose (to find a contradiction) that $a + b \in \Q$. By \lemref{lem:rationals-closed-under-addition-multiplication}:
      \[
        (a + b) - a = b \implies b \in \Q,
      \]
      which is not true, so $a + b \not\in \Q \implies a + b \in \I$.

      Similarly, suppose $ab \in \Q$ whenever $a \neq 0$. By \lemref{lem:rationals-closed-under-addition-multiplication}:
      \[
        ab \left( \frac{1}{a} \right) = b \implies b \in \Q,
      \]
      which contradicts our assumption, so (when $a \neq 0$), $ab \not\in \Q \implies ab \in \I$.
    \end{subproof}

    Take the real numbers $a < b$ to be arbitrary. Suppose, to the contrary, that $\forall t.\ a < t < b \implies t \in \Q$. 
    Let $t' = t - \sqrt{2}$, so by \lemref{lem:rational-irrational-addition-multiplication}, $t' \in \I$ for all $t'$ where 
    $a - \sqrt{2} < t' < b - \sqrt{2}$, which contradicts the result from \probref{prob:density-rationals-in-reals}.

    Therefore, there must exist an irrational number between any two distinct real numbers.

\end{proof}
