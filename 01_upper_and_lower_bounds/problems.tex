\section{Upper and Lower Bounds}

\begin{problem}\label{prob:bounded-convergent}
  Prove that if $a$ is an upper bound for $A$, and if $a$ is also an element of $A$, then it must be that $a = \sup{A}$.
\end{problem}

\begin{proof}
  To establish that $a = \sup{A}$, we must show that:
  \begin{enumerate}
    \item $a$ is an upper bound for $A$, and \label{item:upper-bound}
    \item $a \leq b$ for any upper bound $b$ of $A$. \label{item:least-upper-bound}
  \end{enumerate}

  Note that (\ref{item:upper-bound}) is given by the problem statement, so we
  only need to show (\ref{item:least-upper-bound}). Immediately, $a$ is the
  least upper bound of $A$ because any $b > a$ (whilst being an upper bound of
  $A$) would not be the \textit{least} upper bound (since $a$ is a smaller upper
  bound), and any $b < a$ cannot be a smaller upper bound of $A$ (since $a$ is an
  element of $A$ and thus $b$ would not be greater than or equal to all
  elements of $A$).
\end{proof}


\begin{problem}\label{prob:sup-subset}
  Let $A$ and $B$ be non-empty sets. Prove that if $B \subseteq A$, then $\sup{B} \leq \sup{A}$.
\end{problem}

\begin{proof}
  Let $a = \sup{A}$ and $b = \sup{B}$. We need to show that $b \leq a$.

  Since $B \subseteq A$, every element of $B$ is also an element of $A$. Thus,
  by the definition of supremum, $b$ must be less than or equal to any upper
  bound of $A$. Since $a$ is an upper bound for $A$, we have:
  \[
    b \leq a.
  \]
  Therefore, we conclude that $\sup{B} \leq \sup{A}$.
\end{proof}


\begin{problem}\label{prob:sup-inf-practice}
  Compute, without proof, the supremum and infimum of the following sets:
  \begin{enumerate}[label=(\alph*)]
    \item $\{n \in \N : n^2 < 10\}$
      
    $\sup = 3, \quad \inf = 1$

    \item $\{n/(m + n) : m, n \in \N\}$

    $\sup = 1, \quad \inf = 0$

    \item $\{n/(2n + 1) : n \in \N\}$

    $\sup = 1/2, \quad \inf = 1/3$

    \item $\{n/m : m, n \in \N \text{ with } m + n \leq 10\}$

    $\sup = 9, \quad \inf = 1/9$
  \end{enumerate}
\end{problem}

\begin{problem}\label{prob:sup-inf-linear-transformation}
  Let $A \subseteq \R$ be a nonempty bounded set, and let $c \in \R$. Define the sets $c + A$ and $cA$ by 
  $c + A = \{c + a : a \in A\}$ and $cA = \{ca : a \in A\}$.
  \begin{enumerate}[label=(\alph*)]
    \item Show that $\sup(c + A) = c + \sup A$. \label{prob:sup-c+A}

      \begin{proof}
        Let $s = \sup A$. Since $A$ is nonempty and bounded above, $s$ exists. First we will show that $c + s$ is an upper bound for $c + A$, 
        and then we will show that it is the least upper bound.

        For every $a \in A$, we have $a \leq s$. Therefore, adding $c$ to both sides gives us:
        \[
          c + a \leq c + s,
        \]
        so every element of $c + A$ is less than or equal to $c + s$. This shows that $c + s$ is an upper bound for $c + A$.

        Next, suppose for sake of contradiction that $c + s$ is not the \textit{least} upper bound for $c + A$.
        This would mean $b = \sup(c + A) < c + s$ for some $b \in \R$. By the definition of the supremum, there exists some $a \in A$ such that:
        \[
          b - c < a,
        \] 
        and adding $c$ to both sides gives us:
        \[
          b < a + c,
        \]
        resulting in a contradiction since $c + a \in c + A$ and $b$ is supposed to be an upper bound for $c + A$.
        Therefore, no such $b$ can exist, and we conclude that $c + s$ is indeed the least upper bound for $c + A$.
      \end{proof}

    \item If $c \geq 0$, show that $\sup(cA) = c \sup A$. \label{prob:sup-cA}

      \begin{proof}
        Let $s = \sup A$. Since $A$ is nonempty and bounded above, $s$ exists. We will show that $c s$ is an upper bound for $cA$, and then 
        we will show that it is the least upper bound.

        For every $a \in A$, we have $a \leq s$. Therefore, multiplying both sides by $c \geq 0$ gives us:
        \[
          ca \leq cs,
        \]
        so every element of $cA$ is less than or equal to $cs$. This shows that $cs$ is an upper bound for $cA$. 
        Next, we proceed by casework on the value of $c$ to show that $cs$ is the least upper bound.

        When $c > 0$, let us suppose for sake of contradiction that $cs$ is not the \textit{least} upper bound for $cA$. 
        This means $b = \sup(cA) < cs$ for some $b \in \R$. First, consider when $c > 0$. Dividing through by $c > 0$ gives us:
        \[
          \frac{b}{c} < s,
        \]
        so there exists some $a \in A$ such that:
        \[
          a > \frac{b}{c} \implies ca > b.
        \]
        However, a contradiction is reached because $ca \in cA$ and $b$ is supposed to be an upper bound for $cA$.

        Now consider the case when $c = 0$. In this case, $cA = \{0\}$, and thus $\sup(cA) = 0 = cs$. 

        Therefore, in both cases, we conclude that $cs$ is indeed the least upper bound for $cA$.
      \end{proof}

    \item If $c < 0$, show that $\sup(cA) = c \inf A$. \label{prob:sup-cA-negative}

      \begin{proof}

        Let $c = -d$ where $d > 0$. Then we can rewrite the set $cA$ as $-dA = \{-da : a \in A\} = -(dA)$. The following lemma will be useful:

        \begin{lemma} \label{lem:sup-neg-bounded}
          If $B \subseteq \R$ is nonempty and bounded below, then $\sup(-B) = -\inf B$ where $-B = \{-b : b \in B\}$.
        \end{lemma}

        \begin{subproof}[Proof of \lemref{lem:sup-neg-bounded}]
          Let $m = \inf B$, so for every $b \in B$, we have $b \geq m$. Therefore, multiplying both sides by $-1$ (which reverses the inequality) 
          gives us:
          \[
            -b \leq -m,
          \]
          so every element of $-B$ is less than or equal to $-m$, thus $-m$ is an upper bound for $-B$. Now, suppose $u < -m$ is an upper bound for $-B$. 
          Then,  $-u > m$, which means there exists some $b \in B$ with $m \leq b < -u \implies -b > u$ (by the definition of the infimum), so $u$ cannot be an upper bound for $-B$.

          Hence, $-m = -\inf B$ is the least upper bound for $-B$. 
        \end{subproof}

        Invoking \lemref{lem:sup-neg-bounded} and result of \ref{prob:sup-cA}, we have $\sup(-dA) = -d \inf(A)$, and thus, $\sup(cA) = c \inf(A)$ when $c < 0$. 

      \end{proof}

  \end{enumerate}
\end{problem}

\begin{problem} \label{prob:sup-upper-bound}
  If $\sup A < \sup B$, then show that there exists some $b \in B$ that is an upper bound for $A$.
\end{problem}

\begin{proof}
  Let $s_A = \sup A$ and $s_B = \sup B$. Since $s_A < s_B$, there exists some
  $\eps$ such that $s_B - s_A > \eps > 0$. 

  Notice that there exists some $b \in B$ such that $s_B - \eps < b \leq s_B$
  (by the definition of the supremum) and $s_B - \eps > s_A$, so taken together
  we have some $b \in B$ that is an upper bound for $A$ because for every $a \in A$, we have
  \[
    a \leq s_A < s_B - \eps < b,
  \]
  and therefore, some $b \in B$ is an upper bound for $A$.
\end{proof}
