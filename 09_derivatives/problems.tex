\section{Derivatives}

\begin{callout}
  We take the following theorem to be given (as the proof has been
  \href{https://youtu.be/Av210fDPukY}{provided}).
  \begin{theorem}[Product Rule for Derivatives]
  \label{thm:prod-rule-deriv}
    Let $f$ and $g$ be differentiable functions. Then the product $f \cdot g$ is differentiable, and
    \[
      (f \cdot g)'(x) = f'(x) \cdot g(x) + f(x) \cdot g'(x).
    \]
  \end{theorem}

\end{callout}

\begin{problem}
  Using the definition of the derivative, compute $f'(x)$ for $f(x) = 3x + 5$.
  \vspace{\baselineskip}

  By the definition of the derivative:
    \begin{align*}
    f'(x) &= \lim_{h \to 0}\frac{f(x+h)-f(x)}{h}\\
          &= \lim_{h \to 0}\frac{3(x+h)+5-(3x+5)}{h}\\
          &= \lim_{h \to 0}\frac{3x+3h+5-3x-5}{h}\\
          &= \lim_{h \to 0}\frac{3h}{h}\\
          &= \lim_{h \to 0}3\\
          &= 3.
    \end{align*}
\end{problem}

\begin{problem}
  Find the derivative of $f(x) = x^{4} - 2x^{3} + 7x - 1$.
  \vspace{\baselineskip}

  By the definition of the derivative, we have
    \begin{align*}
      f'(x) &= \lim_{h \to 0}\frac{f(x+h)-f(x)}{h}\\
            &= \lim_{h \to 0}\frac{(x+h)^4-2(x+h)^3+7(x+h)-1 - (x^4-2x^3+7x-1)}{h}. \\
    \end{align*}
  Expanding, using the binomial theorem, the higher degree terms produces
  \[
    (x+h)^4 = x^4 + 4x^3h + 6x^2h^2 + 4xh^3 + h^4, \quad
    (x+h)^3 = x^3 + 3x^2h + 3xh^2 + h^3.
  \]
  Substituting, simplifying, and evaluating the limit gives us the desired result:
  \[
    f'(x) = 4x^3 - 6x^2 + 7.
  \]
\end{problem}

\begin{problem}
  Compute $\deriv{}{x} \left( \dfrac{1}{x^{2}} \right)$.
  \vspace{\baselineskip}

  Let $f(x) = \left( \dfrac{1}{x^{2}} \right)$. Using the definition of the derivative:
    \begin{align*}
      f'(x) &= \lim_{h \to 0}\frac{\frac{1}{(x+h)^2} - \frac{1}{x^2}}{h}\\
            &= \lim_{h \to 0}\frac{x^2 - (x+h)^2}{h(x+h)^2x^2}\\
            &= \lim_{h \to 0}\frac{x^2 - (x^2 + 2xh + h^2)}{h(x+h)^2x^2}\\
            &= \lim_{h \to 0}\frac{-2xh - h^2}{h(x+h)^2x^2}\\
            &= \lim_{h \to 0}\frac{-2x - h}{(x+h)^2x^2}\\
            &= -\frac{2}{x^3}.
    \end{align*}
\end{problem}

\begin{problem}
  Using the definition of the derivative, show that if $f(x) = c$ (a constant), then $f'(0) = 0$.
  \vspace{\baselineskip}

  By the definition of the derivative:
    \begin{align*}
      f'(x) &= \lim_{h \to 0}\frac{f(x+h)-f(x)}{h}\\
            &= \lim_{h \to 0}\frac{c-c}{h} = \lim_{h \to 0}\frac{0}{h}\\
            &= 0.
    \end{align*}
\end{problem}

\begin{problem}
  Use the product rule to find the derivative of $f(x) = (x^{2} + 1)(x^{3} - 2x)$.
  \vspace{\baselineskip}

  Let $u(x) = x^2 + 1$ and $v(x) = x^3 - 2x$. First, we compute the derivatives:
  \[
    u'(x) = 2x, \quad v'(x) = 3x^2 - 2.
  \]
  Next, we substitute into \thmref{thm:prod-rule-deriv}:
    \begin{align*}
    f'(x) &= u'(x)v(x) + u(x)v'(x)\\
          &= 2x(x^3 - 2x) + (x^2 + 1)(3x^2 - 2)\\
          &= 2x^4 - 4x^2 + 3x^4 + 3x^2 - 2x^2 - 2\\
          &= 5x^4 - 3x^2 - 2.
    \end{align*}
\end{problem}

\begin{problem}
  Compute the derivative of $f(x) = \sqrt{x}$ at $x = 9$ using the definition of the derivative.
  \vspace{\baselineskip}

  Let $f(x) = \sqrt{x}$. Then, by the definition of the derivative:
    \begin{align*}
      f'(x) &= \lim_{h \to 0}\frac{\sqrt{x+h} - \sqrt{x}}{h}\\
            &= \lim_{h \to 0}\frac{(\sqrt{x+h} - \sqrt{x})(\sqrt{x+h} + \sqrt{x})}{h(\sqrt{x+h} - \sqrt{x})}\\
            &= \lim_{h \to 0}\frac{(x + h) - x}{h(\sqrt{x+h} - \sqrt{x})}\\
            &= \frac{1}{2 \sqrt{x}}.
    \end{align*}
\end{problem}

\begin{problem}
  If $f(x) = x^{n}$ where $n$ is a positive integer, prove (using the definition of the derivative) that 
  $f'(x) = nx^{n-1}$.

  \begin{proof}
    By the definition of the limit and the binomial theorem, we have:
    \begin{align*}
       \deriv{}{x} x^n = \lim_{h \to 0} \frac{(x + h)^{n} - x^{n}}{h} &= \lim_{h \to \infty} \frac{\sum_{k=0}^{n}\binom{n}{k}x^{k}h^{n-k}}{h}\\
                                                     &= \lim_{h \to \infty} \sum_{k=0}^{n-1}\binom{n}{k}x^{k}h^{n-k-1}\\
                                                     &= nx^{n-1}.
    \end{align*}
  \end{proof}
\end{problem}

\begin{problem}
  Consider the function
    \begin{equation*}
      f(x) = 
      \begin{cases}
        x^2 \sin(1/x) & \text{if } x \neq 0 \\
        0 & \text{if } x = 0
      \end{cases}
    \end{equation*}
  Show that $f$ is differentiable at $x = 0$, but $f'$ is not continuous at $x = 0$.

  \begin{proof}
    We will first determine how $f'$ is defined. As a consequence of this definition, we will be able to establish
    that $f'(0)$ is well-defined. We then will demonstrate that $f'$ is not continuous.
    (Before we proceed, we will prove the analogue of \thmref{thm:squeeze-theorem-sequences} for 
    functional limits.)

    \begin{theorem}[Squeeze Theorem for Functional Limits]
      \label{thm:squeeze-theorem-fn-limits}
      Let $f, g, h$ satisfy $f(x) \leq g(x) \leq h(x)$ for all $x$ in some common domain $A$. If
      \[
        \lim_{x \to c} f(x) = L = \lim_{x \to c} h(x)
      \]
      at some limit point $c \in A$, then
      \[
        \lim_{x \to c} g(x) = L.
      \]
    \end{theorem}

    \begin{subproof}
      Let $(x_{n})$ be an arbitrary sequence taken from $A$ that converges to
      $c$ with $x_{n} \neq c$ for any $n \geq 1$. Then, by the Order Limit
      Theorem, \thmref{thm:squeeze-theorem-sequences}, and the hypothesis, we
      have
      \[
        L = \lim_{n \to \infty} f(x_{n}) \leq \lim_{n \to \infty} g(x_{n}) \leq \lim_{n \to \infty} h(x_{n}) = L,
      \]
      therefore $\lim\limits_{n \to \infty} g(x_{n}) = L$. Since $(x_{n})$ was
      an arbitrary sequence tending to $c$, by the definition of a functional limit,
      we conclude:
      \[
        \lim_{x \to c} g(x) = L.
      \]
    \end{subproof}

    Returning to showing that $f'(0)$ exists, we can leverage the limit
    definition of the derivative to evaluate $f'(0)$:
    \[
      f'(0) = \lim_{h \to 0} \frac{f(0+h) - f(0)}{h} = \lim_{h \to 0} \frac{h^2 \sin \left( \frac{1}{h} \right) - 0}{h} = \lim_{h \to 0} h \sin \left( \frac{1}{h} \right).
    \]
    Since $\abs{\sin \left( \frac{1}{h} \right)} \leq 1$ for all $h \neq 0$,
    \[
      0 \leq \abs{h \sin \left( \frac{1}{h} \right)} \leq \abs{h}.
    \]
    As $h \to 0$, $\abs{h} \to 0$. Therefore, by \thmref{thm:squeeze-theorem-fn-limits},
    \[
      f'(0) = \lim_{h \to 0} h \sin\left(\frac{1}{h}\right) = 0.
    \]
    Hence, $f$ is differentiable at $x = 0$.

    To show that $f'$ is not continuous everywhere, it is sufficient to prove
    that $f'$  is not continuous at $x = 0$.

  \end{proof}
\end{problem}

\begin{problem}
  % TODO: Prove the power rule for negative exponents, chain rule first
  Let $f$ and $g$ be differentiable functions. Prove the quotient rule:
    \begin{equation*}
      \left(\frac{f}{g}\right)' (x) = \frac{f'(x)g(x) - f(x)g'(x)}{[g(x)]^2}
    \end{equation*}
  provided $g(x) \neq 0$.

  \begin{proof}
    Notice that $\frac{f(x)}{g(x)} = f(x) \cdot g(x)^{-1}$.
    Let $h(x) = g(x)^{-1}$. By the \thmref{thm:prod-rule-deriv}, chain rule, and the power
    rule, we have:
    \begin{align*}
        \deriv{}{x} \left( f(x) \cdot {h(x)} \right) &= f'(x)h(x) + h'(x)f(x)\\
                                                    &= \frac{f'(x)}{g(x)} -\frac{g'(x) f(x)}{g(x)^{2}}\\
                                                    &= \frac{f'(x) g(x) - f(x) g'(x)} {g(x)^2}.
    \end{align*}
  \end{proof}

\end{problem}

\begin{problem}
  Suppose $f$ has a local maximum at $x = a$ and $f$ is differentiable at $x = a$.
  Prove that $f'(a) = 0$. (This is Fermat's theorem.)
\end{problem}
