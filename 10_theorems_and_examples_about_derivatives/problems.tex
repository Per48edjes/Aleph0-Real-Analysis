\section{Theorems / Examples about Derivatives}

\setcounter{problem}{6}

\begin{callout}
  The first several problems for this unit were derivative calculations with multiple choice
  answers and are omitted here.
\end{callout}

\begin{problem}
  \begin{enumerate}[label=(\alph*)]
    \item Prove that if $f$ is differentiable at $a$ and $f(a) \neq 0$, then $\abs{f}$
      is differentiable at $a$.

      \begin{proof}
        Let $g(x) = \abs{f(x)}$. Then, by \thmref{thm:chain-rule}, we have
        \[
          g'(a) = 
          \begin{cases}
            -f'(a) & \text{if } f(a) < 0 \\
            f(a) & \text{if } f(a) > 0,
          \end{cases}
        \]
        because $g$ is differentiable at $f(a) \neq 0$ (since the absolute
        value function is differentiable everywhere except $0$) and $f$ is
        differentiable at $a$ (by the hypothesis). Thus, $\abs{f}$ is
        differentiable at $a$ so long as $f(a) \neq 0$.
      \end{proof}

    \item Give a counterexample if $f(a) = 0$.
      \vspace{\baselineskip}

      Consider $f(x) = x$ and let $a = 0$. Clearly, $f(a) = 0$ and $\abs{f}$ is not differentiable at $a$.
  \end{enumerate}
\end{problem}

\begin{problem}
  Show that a number $a$ is a double root of a polynomial $f$ if and only if $a$ is a root of both $f$ and $f'$.

  \begin{proof}
    We assume that $f$ and $f'$ are polynomials in $x$ and that $a$ is a root of $f$.
    \vspace{\baselineskip}

    \begin{forwardimplication}
      Suppose $a$ is a double root of $f$. Then, we may express $f(x) = (x - a)^{2}P(x)$ where $P$ is any polynomial in $x$.
      Trivially, $a$ is a root of $f$. \thmref{thm:prod-rule-deriv} gives
      \[
        f'(x) = 2(x - a)P(x) + P'(x)(x - a)^{2},
      \]
      a polynomial having $a$ as a root as well.
    \end{forwardimplication}
    \vspace{\baselineskip}

    \begin{backwardimplication}
      Suppose $a$ is a root of $f$ and $f'$. Then, by \thmref{thm:prod-rule-deriv},
      \[
        f(x) = (x-a)P(x) \implies f'(x) = (x-a)P'(x) + P(x),
      \]
      for some polynomial $P$ in $x$.
      Since $f'(a) = 0$, it must be the case that $(x - a)$ divides $P(x)$. Let
      $Q(x) = P(x) / (x-a)$. We can rewrite $f$:
      \[
        f(x) = (x-a)^{2}Q(x).
      \]
      Hence, $a$ is a double root of $f$.
    \end{backwardimplication}
  \end{proof}
\end{problem}
