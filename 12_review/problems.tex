\section{Review}

\begin{problem}
  Show that it is impossible to write $x = f(x)g(x)$ with $f, g$ differentiable and $f(0) = g(0) = 0$.

  \begin{proof}
    Supposing it were possible to write $x = f(x)g(x)$ with $f, g$ differentiable and $f(0) = g(0) = 0$,
    by \thmref{thm:prod-rule-deriv} we have
    \[
      x = f(x)g(x) \implies 1 = f'(x)g(x) + f(x)g'(x).
    \]
    When evaluated at $x = 0$, we obtain $1 = f'(x) \cdot 0 + 0 \cdot g'(x) = 0$ -- an impossibility.
  \end{proof}
\end{problem}

\begin{problem}
  For all $x, y$ in the real numbers, use \thmref{thm:mean-value-theorem} to show that
  \[
    \abs{\sin{x} - \sin{y}} \leq \abs{x - y}.
  \]

  \begin{proof}
    Let $x, y \in \R$ be arbitrary and, without loss of generality, $x > y$. Also, let $f(x) = \sin{x}$.
    Then, by \thmref{thm:mean-value-theorem}, we see that there exists some $c \in (y, x)$
    such that
    \[
      f'(c) = \frac{f(x) - f(y)}{x - y}.
    \]
    Taking absolute values, it is clear that 
    \[
      \abs{f'(c)} = \abs{\cos{c}} = \frac{\abs{f(x) - f(y)}}{\abs{x - y}} \leq 1 \implies \abs{\sin{x} - \sin{y}} \leq \abs{x - y}.
    \]
  \end{proof}
\end{problem}

\begin{problem}
  Let $a \geq 0$ and define
  \[
    f_{a}(x) =
    \begin{cases}
      x^{a} &\text{ if } x > 0\\
      0     &\text{ if } x \leq 0.
    \end{cases}
  \]
  Show that:
  \begin{enumerate}[label=(\roman*)]
    \item $f_{a}$ is continuous at $0$ if $a > 0$. \label{prob:real-positive-powers-rhs-limit}
      \begin{proof}
        We must show that $\lim\limits_{x \to 0} f_{a}(x) = f_{a}(0) = 0$. It is clear that
        $\lim\limits_{x \to 0^{-}} f_{a}(x) = 0$, so it is sufficient to show
        that $\lim\limits_{x \to 0^{+}} f_{a}(x) = 0$. Notice that we can write
        $x^{a} = e^{a \ln{x}}$ since $a$ is a positive real number. Now, let
        $\eps > 0$ be given. Put $\delta = e^{(\ln{\eps})/a}$.
        Then, we have
        \[
          \forall x > 0.\, x < \delta \implies e^{a \ln{x}} < \eps,
        \]
        thus the right-sided limit equals $0$ as desired.
      \end{proof}

    \item $f_{a}$ is differentiable at $0$ if $a > 1$.
      \begin{proof}
        Since the left-sided limit of the difference quotient is obviously $0$,
        we restrict our attention to the right-sided limit to show:
        \[
          \lim_{h \to 0^{+}} \frac{f_{a}(h)}{h} = \lim_{h \to 0^{+}} \frac{h^{a}}{h} = \lim_{h \to 0^{+}} h^{a - 1} = 0.
        \]
        Let $b = a - 1$; since $a > 1$, $b > 0$. Using the result from
        \probref{prob:real-positive-powers-rhs-limit}, we verify that
        \[
          \lim_{h \to 0^{+}} h^{b} = 0.
        \]
      \end{proof}
  \end{enumerate}
\end{problem}


\begin{problem}
  Show that if $g$ is continuous at $0$, then $f(x) = xg(x)$ is differentiable at $0$ with $f'(0) = g(0)$.

  \begin{proof}
    To show the consequent, we must demonstrate that
    \[
      \lim_{h \to 0} \frac{(0 + h)g(0 + h)}{h} = g(0).
    \]
    Simplifying and invoking the hypothesis we do, indeed, have
    \[
      \lim_{h \to 0} g(h) = g(0).
    \]
  \end{proof}
\end{problem}
