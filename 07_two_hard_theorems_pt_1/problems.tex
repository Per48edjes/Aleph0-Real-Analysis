\section{Two Hard Theorems (Part 1)}

\begin{problem}
  Is there a continuous function $f : \R \to \R$ where $f(\R) = \Q$? What about $f(\R) = \Z$?

  \vspace{\baselineskip}

  No such function exists for either case.

  \begin{proof}
    Suppose $f : \R \to \R$ is continuous and $f(\R) \subseteq \Q$. 

    If $f$ is not constant, then there exist $a<b$ with $f(a) \neq f(b)$. By
    \thmref{thm:intermediate-value-theorem}, $f([a,b])$ contains every real
    number between $f(a)$ and $f(b)$. But any such interval contains irrational
    values, contradicting $f([a,b]) \subseteq \Q$. Therefore $f$ must be
    constant, so $f(\R)$ is a singleton. This cannot equal $\Q$, hence no such
    function exists.

    The same reasoning applies if we assume $f(\R) \subseteq \Z$.
  \end{proof}

\end{problem}

\begin{problem}
  Show the following more general consequence of the IVT:

  Suppose $f: \R \to \R$ is continuous, and $f(a) < f(b)$ for some $a < b$. Let
  $u \in (f(a), f(b))$ be a real number. Then there exists a real number $c \in
  (a, b)$ such that $f(c) = u$.

  \begin{proof} 
    See proof of \thmref{thm:intermediate-value-theorem}.
  \end{proof} 
\end{problem}

\begin{problem}
  Let $f : \R \to \R$ be continuous. Suppose that $f([0, 1])$ is contained in $[0, 1]$. Prove that 
  $f$ must have a fixed point; that is, show thaat $f(x) = x$ for at least one value of $x \in [0, 1]$.

  \begin{proof}
    Define $g(x) = f(x) - x$, which is continuous on $[0,1]$.
    Since $f([0,1]) \subseteq [0,1]$, we have
    \[
      g(0) = f(0) - 0 \geq 0, \qquad g(1) = f(1) - 1 \leq 0.
    \]
    If either $g(0) = 0$ or $g(1) = 0$, then $f$ has a fixed point at $0$ or $1$, respectively.
    Otherwise, $g(0) > 0$ and $g(1) < 0$. By \thmref{thm:intermediate-value-theorem},
    there exists $x \in (0,1)$ with $g(x) = 0$, so $f(x) = x$.
    Thus, $f$ has a fixed point in $[0,1]$.
  \end{proof}
\end{problem}

\begin{problem}
  A function $f : \R \to \R$ is increasing if $f(x) \leq f(y)$ for all $x < y$. Show the following converse to the IVT:

  \begin{quote}
    Suppose that $f$ is increasing and satisfies the conclusion of the IVT
    (i.e., for all $x < y$ and all $L$ between $f(x)$ and $f(y)$, there exists
    $c \in (x, y)$ where $f(c) = L$. Then, $f$ is continuous.
  \end{quote}

  \begin{proof}
    We prove the contrapositive. Suppose $f$ is increasing but not continuous
    at some $b \in \R$. We show that the conclusion of the IVT does not hold.

    Define
    \[
      l = \sup \{ f(x) : x < b \}, \qquad u = \inf \{ f(x) : x > b \}.
    \]
    (These sets are well defined since $f$ is increasing, hence, the sets are
    bounded above and below, respectively).
    Since $f$ is not continuous at $b$, we must have $l<u$.
    Let $L = \dfrac{l+u}{2}$, so $l < L < u$.

    Since $l < L < u$ and $L - l = u - L$, set $\eps = \frac{L - l}{2} > 0$.
    By the definition of supremum, there exists $x_{1} < b$ such that
    \[
      l - \eps < f(x_{1}) \le l < l + \eps < L.
    \]
    Similarly, by the definition of infimum, there exists $x_{2} > b$ such that
    \[
      L < u - \eps < u \leq f(x_{2}) < u + \eps.
    \]
    Thus, $f(x_{1}) < L < f(x_{2})$.

    Now, for any $x'\in(x_{1},x_{2})$, if $x' < b$ then $f(x') \leq l < L$, and if
    $x' > b$ then $f(x') \geq u>L$. Hence no $x' \in (x_{1}, x_{2})$ satisfies
    $f(x') = L$.
    Therefore, the IVT conclusion fails, which completes the proof of the contrapositive.
  \end{proof}

\end{problem}

\begin{problem}
  Imagine a clock where the hour hand and the minute hand are indistinguishable
  from each other. Assuming the hands move continuously around the face of the
  clock, and assuming their positions can be measured with perfect accuracy, is
  it always possible to determine the time?

  \vspace{\baselineskip}

  No, it is not always possible to determine the time.
\end{problem}
