\section{Two Hard Theorems (Part 2)}


\begin{callout}
  We take the following theorem to be given (as the proof has been
  \href{https://youtu.be/39ntUHbmXpc}{provided}).
  \begin{theorem}[Extreme Value Theorem]
  \label{thm:extreme-value-theorem}
    Suppose that $f : [a, b] \to \R$ is continuous. Then there exist points $c, d \in [a, b]$ such that
    \[
      f(c) \leq f(x) \leq f(d)
    \]
    for all $x \in [a, b]$. In other words, $f$ attains its minimum value at
    $c$ and its maximum value at $d$.
  \end{theorem}
\end{callout}

\begin{problem}
  Give a counterexample to \thmref{thm:extreme-value-theorem} on the open
  interval $(a, b)$ and on the unbounded domain $\R$. 
  
  \vspace{\baselineskip}

  First, consider $f = \ln(x - a)$ on $(a, b)$.  Clearly, $f$ is
  continuous, however, it is unbounded below, so $f$ never attains a minimum on
  $(a, b)$.
  
  Second, let $g : \R \to \R$ with $g(x) = x$. While $g$ is continuous everywhere, it is
  unbounded below, so $g$ never attains a minimum on $\R$. Furthermore, $g$
  never attains a maximum on $\R$.
\end{problem}

\begin{problem}
  Show that if $f$ is continuous on $[a,b]$ with $f(x) > 0$ for all $a \leq x \leq b$,
  then $\dfrac{1}{f}$ is bounded on $[a,b]$. 

  \begin{proof}
    From \thmref{thm:extreme-value-theorem}, we know that $f$ attains a maximum
    $M > 0$ and a minimum $m > 0$ on $[a,b]$. Then, $\dfrac{1}{f}$ is bounded
    below by $1/M$ and above by $1/m$.
  \end{proof}

\end{problem}

\begin{problem}
  In the video, we showed the ``maximum'' statement of
  \thmref{thm:extreme-value-theorem}. From this ``maximum'' statement, deduce
  the ``minimum'' statement (i.e., if $f : [a,b] \to \R$ is continuous, then
  $f$ attains a minimum value). 

  \begin{proof}
    We will show that if $f$ is continuous on $[a,b]$, then $f$ attains a minimum
    value on $[a,b]$.
    First, define $g(x) = -f(x)$. Then by the ``maximum'' statement of \thmref{thm:extreme-value-theorem},
    $g$ attains a maximum value at some point $c \in [a,b]$. Then $f(c) \leq f(x)$ for all $x \in [a,b]$.
  \end{proof}

\end{problem}

\begin{problem}
  Which part (or parts) of the proof of \thmref{thm:extreme-value-theorem} fails for open intervals $(a, b)$?
\end{problem}

\begin{problem}
  Which part (or parts) of the proof of \thmref{thm:extreme-value-theorem} fails for unbounded intervals? 
\end{problem}

\begin{problem}
  Let $f : [a, b] \to \R$ be continuous and fix a point $y \in \R$. Show that there exists $m \geq 0$ such that
  $\abs{f(x) - y} \geq m$ for all $x \in [a, b]$.

  \begin{proof}
  \end{proof}
\end{problem}
