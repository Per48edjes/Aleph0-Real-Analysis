\section{Miscellaneous Problems}

\begin{problem}
  Let $P_{n} = \{ 1, 2, \dots, n \}$, and for any nonempty subset $S \subseteq P_{n}$, define
  \[
    v(S) = \dfrac{1}{|S|} \sum_{s \in S} s. 
  \]
  Also, define 
  \[
    \Scal_{n} = \{ v(A) - v(B) : A \cup B = P_n,\ A \cap B = \emptyset,\ A, B \neq \emptyset \},
  \]
  and 
  \[
    \diam{\Scal_{n}} = \sup \Scal_{n} - \inf \Scal_{n}.
  \]
  Show that $\lim\limits_{n \to \infty} \frac{\diam{\Scal_{n}}}{n} = 1$.

  \begin{callout}
    This is a slightly modified version of the problem posted by the user \texttt{chris\_34203}
    on the \href{https://discord.com/channels/1381996954237800608/1381996954673745932/1393174128432906357}{Aleph 0 Math Membership} Discord server.
  \end{callout}

\end{problem}

\begin{proof}
  We will show $\diam{\Scal_{n}} = n$ for all $n \ge 2$, which immediately implies the limit.

  First, notice the symmetry inherent in $\diam{\Scal_{n}}$: swapping $A$ and $B$ in any partition negates $v(A)-v(B)$, so
  \[
    \sup \Scal_{n} = -\inf \Scal_{n} \implies \diam{\Scal_{n}} = 2\sup \Scal_{n}.
  \]

  Next, we will bound $\sup \Scal_{n}$. Taking $A = \{ n \}$ and $B = P_n \setminus \{ n \}$ gives
  \[
    v(A) = n, \quad v(B) = \frac{1 + 2 + \cdots + (n-1)}{n - 1} = \frac{n}{2},
  \]
  which provides a lower bound for $\sup \Scal_{n}$ since $A, B$ were fixed:
  \[
    v(A) - v(B) = \frac{n}{2} \implies \sup \Scal_{n} \ge \frac{n}{2}.
  \]
  To show $\sup \Scal_{n} \le \frac{n}{2}$, choose any disjoint partition $A \cup B = P_{n}$ with $|A| = k$, so $1 \le k \le n - 1$.
  Let $S_A = \sum\limits_{a \in A} a$ and $S_B = \sum\limits_{b \in B} b$. Then,
  \[
    S_{A} \le n + (n-1) + \cdots + (n - k + 1) = \frac{k(2n - k + 1)}{2},
  \]
  and
  \[
    S_{B} \ge 1 + 2 + \cdots + (n - k) = \frac{(n - k)(n - k + 1)}{2}.
  \]
  Therefore, we have the desired upper bound for $\sup \Scal_{n}$ since $A, B$ were arbitrary:
  \[
    v(A) - v(B) = \frac{S_A}{k} - \frac{S_B}{n - k} \le \frac{2n - k + 1}{2} - \frac{n - k + 1}{2} = \frac{n}{2} \implies \sup \Scal_{n} \le \frac{n}{2}.
  \]

  Thus, we have shown $\frac{n}{2} \leq \sup \Scal_{n} \leq \frac{n}{2} \implies \diam{\Scal_{n}} = n$, and consequently,
  the limit follows.
\end{proof}

\begin{problem}
  
  If $\lim\limits_{n \to \infty} a_n = L > 0$, then show that $\lim\limits_{n \to \infty} \sqrt{a_n} = \sqrt{L}$. Be sure to discuss
  the issue when $\sqrt{a_{n}}$ makes sense.

  \begin{proof}
    Since $a_{n} \to L > 0$, there exists $N' \in \N$ such that
    $a_{n} \ge 0$ for all $n > N'$.  Hence every $\sqrt{a_{n}}$ that appears
    below is real, and $\sqrt{L}$ is real because $L > 0$.

    Let $\eps > 0$ be given. By convergence of $a_{n}$, choose $N \ge N'$ such that
    for all $n > N$,
    \[
      \abs{a_{n} - L}< \eps^{2}.
    \]
    Notice that 
    \begin{align*}
      \abs{a_{n} - L} &= \abs{(\sqrt{a_n} - \sqrt{L})(\sqrt{a_n} + \sqrt{L})} \\
                      &= \abs{\sqrt{a_n} - \sqrt{L}}\abs{\sqrt{a_n} + \sqrt{L}} \quad < \eps^{2},
    \end{align*}
    and that
    \[
      \abs{\sqrt{a_{n}} + \sqrt{L}} \geq \abs{\sqrt{a_n} - \sqrt{L}}.
    \]
    Taken together
    \[
      \abs{\sqrt{a_{n}} - \sqrt{L}}^{2} \leq \abs{\sqrt{a_{n}} - \sqrt{L}}\abs{\sqrt{a_{n}} + \sqrt{L}} < \eps^{2},
    \]
    which implies
    \[
      \abs{\sqrt{a_{n}} - \sqrt{L}} < \eps.
    \]

    Because $\eps > 0$ was arbitrary, we conclude
    \[
      \lim_{n \to \infty} \sqrt{a_{n}} = \sqrt{L}.
    \]
  \end{proof}

\end{problem}

\begin{problem}
  Suppose functions $f : \R \to \R$ and $g : \R \to \R$ are continuous and $f(p) = g(p)$ for all all $p \in \Q$. Prove that
  $f = g$.

  \begin{callout}
    This problem comes from the \texttt{Fee Phi Fo Fum} \href{https://www.youtube.com/watch?v=OuwLcZQsDoY}{YouTube channel}.
  \end{callout}

  \begin{proof}
    We already know $f = g$ over the rationals, so it is sufficient to show $f = g$ for all irrational numbers.

    Let $x \in \I$ be arbitrary. Consider the sequence $(a_{n})_{n=1}^{\infty} = 1/n$. Because
    $\Q$ is dense in $\R$ (by the result of \probref{prob:infinite-rationals-in-reals}), we have
    \[
      \forall n \in \N.\, \exists q_{n} \in \Q.\, x < q_{n} < x + a_{n},
    \]
    and $(q_{n}) \to x$ as $n \to \infty$. By continuity, we obtain
    \[
      f(x) = \lim_{n \to \infty} f(q_{n}) = \lim_{n \to \infty} g(q_{n}) = g(x).
    \]
    Thus, $f(x) = g(x)$ for all $x \in \I$ (since $x \in \I$ was arbitrary), and $f = g$ for all real numbers.
  \end{proof}
\end{problem}

\begin{problem}
  \leavevmode\par\noindent

  \begin{callout}
    This is Exercise 2.7.9 from \textit{Understanding Analysis} (Abbott 2015).
  \end{callout}

  \begin{definition}
    Given a series $\sum_{n=1}^{\infty} a_{n}$ with $a_{n} \geq 0$, the Ratio Test states that if $(a_{n})$ 
    satisfies
    \[
      \lim \abs{\frac{a_{n+1}}{a_{n}}} = r < 1
    \]
    then the series converges absolutely.
  \end{definition}

  \begin{enumerate}[label=(\alph*)]
    \item Let $r'$ satisfy $r < r' < 1$. Explain why there exists an $N$ such that $n \geq N$ implies 
      $\abs{a_{n+1}} \leq \abs{a_{n}}r'$.
      \vspace{\baselineskip}
      
      Set $\eps = r' - r$. Then, by the definition of the limit, there exists some $N \in \N$ such that,
      for all $n \geq N$,
      \[
        \abs{\frac{\abs{a_{n+1}}}{\abs{a_{n}}} - r} < \eps \implies \frac{\abs{a_{n+1}}}{\abs{a_{n}}} < (r' - r) + r \implies \abs{a_{n+1}} \leq \abs{a_{n}} r'
      \]

    \item Why does $\abs{a_{N}} \sum{(r')^{n}}$ converge?
      \vspace{\baselineskip}

      $\abs{a_{N}} \sum{(r')^{n}}$ is a geometric series with common ratio $\abs{r'} < 1$.

    \item Now, show that $\sum{\abs{a_{n}}}$ converges, and conclude that $\sum{a_{n}}$ converges.
      \vspace{\baselineskip}

      By comparison, we see that
      \[
        \sum{\abs{a_{n}}} < \abs{a_{N}} \sum{(r')^{n}}
      \]
      for $n \geq N$, so it follows that $\sum{\abs{a_{n}}}$ converges,
      therefore $\sum{a_{n}}$ converges (since absolute convergence implies
      convergence).
  \end{enumerate}

\end{problem}

\begin{problem}
  \leavevmode\par\noindent

  \begin{callout}
    This is Exercise 2.7.10 from \textit{Understanding Analysis} (Abbott 2015), which relies on the results
    of Exercise 2.4.10, provided as a lemma below:

    \begin{lemma}
      \label{lem:2.4.10}
      $\prod_{n=1}^{\infty} (1 + a_{n})$ with $a_{n} \geq 0$ converges if and only if $\sum a_{n}$ converges.
    \end{lemma}

  \end{callout}

  \begin{enumerate}[label=(\alph*)]
    \item Does $\frac{2}{1} \cdot \frac{3}{2} \cdot \frac{5}{4} \cdot \frac{9}{8} \cdot \frac{17}{16} \cdots$ converge?
      \vspace{\baselineskip}

      Yes.

      \begin{proof}
        Put $a_{n} = 2^{-(n-1)}$. It is sufficient to show that $\sum a_{n}$ converges by \lemref{lem:2.4.10}. 
        Notice that $\sum a_{n}$ is a geometric series with common ratio $r = \frac{1}{2}$:
        \[
          \sum_{n=1}^{\infty} a_{n} = \sum_{n=1}^{\infty} \left( \frac{1}{2} \right)^{n-1} = \frac{1}{1 - \frac{1}{2}} = 2.
        \]
        Since $\sum a_{n}$ converges, $\frac{2}{1} \cdot \frac{3}{2} \cdot \frac{5}{4} \cdot \frac{9}{8} \cdot \frac{17}{16} \cdots$ converges.
      \end{proof}

    \item The infinite product $\frac{1}{2} \cdot \frac{3}{4} \cdot \frac{5}{6} \cdot \frac{7}{8} \cdot \frac{9}{10} \cdots$ certainly converges. (Why?) Does it converge to zero?
      \vspace{\baselineskip}

      The infinite product converges due to \thmref{thm:monotone-convergence}:
      the sequence formed by the partial products is decreasing and this
      sequence is bounded below by $0$. We claim that this infinite product converges to $0$.

      \begin{proof}
        Let $p_{n} = \prod_{k=1}^{n} \left(1 - \frac{1}{2k}\right)$ and $q_{n} = \prod_{k=1}^{n} \left(1 - \frac{1}{2k+1}\right)$.
        Then, we obtain
        \[
          \forall n \in \N. \quad \frac{1}{2n + 1} = p_{n}q_{n} > p_{n}^{2}.
        \]
        Since $\left(\sqrt{ p_{n} q_{n} }\right) = \left( \frac{1}{\sqrt{2n + 1}} \right) \to 0$ as $n \to \infty$, we have 
        $\lim\limits_{n \to \infty} (p_{n}) = 0$ by \thmref{thm:squeeze-theorem-sequences}.
      \end{proof}


    \item In 1655, John Wallis famously derived
      \[
        \left( \frac{2 \cdot 2}{1 \cdot 3} \right) \left( \frac{4 \cdot 4}{3 \cdot 5} \right) \left( \frac{6 \cdot 6}{5 \cdot 7} \right) \left( \frac{8 \cdot 8}{7 \cdot 9} \right) \cdots = \frac{\pi}{2}
      \]
      Show that the left side of this identity at least converges to something.

      \begin{proof}
        Let $\prod_{n=1}^{\infty} (1 + a_{n}) = \frac{2 \cdot 2}{1 \cdot 3} \cdot \frac{4 \cdot 4}{3 \cdot 5} \cdot \frac{6 \cdot 6}{5 \cdot 7} \cdot \frac{8 \cdot 8}{7 \cdot 9} \cdots$, so $a_{n} = (4n^{2} -1)^{-1}$.
        We will show that $a_{n}$ converges, which is sufficient by \lemref{lem:2.4.10}.
        By comparison, we see that
        \[
          \sum_{n=1}^{m} a_{n} < \sum_{n=1}^{m} n^{-2}
        \]
        for all $m \in \N$.
        Since $\sum n^{-2}$ converges, so too does $\sum a_{n}$.
      \end{proof}

  \end{enumerate}

\end{problem}

\begin{problem}
  Prove that $\displaystyle \sum_{n=1}^{\infty} \frac{1}{n^{p}}$ converges if and only if $p > 1$.

  \leavevmode\par\noindent

  \begin{callout}
    This is Exercise 2.7.5 from \textit{Understanding Analysis} (Abbott 2015).
  \end{callout}

  \begin{proof}

    We take the following lemma to be given, which will be useful later in the proof:
    \begin{lemma}
      \label{lem:2.4.6}
      Suppose $(b_{n})$ is decreasing and satisfies $(b_{n}) \geq 0$ for all $n \in \N$. Then, the series
      $\sum_{n=1}^{\infty} b_{n}$ converges if and only if the series
      \[
        \sum_{n=0}^{\infty} = b_{1} + 2b_{2} + 4b_{4} + 8b_{8} + \cdots
      \]
      converges.
    \end{lemma}

    \begin{forwardimplication}
      We will show the contrapositive. Suppose $p \le 1$.
      When $p = 1$ we obtain the harmonic series, which diverges.
      For any $p < 1$, we have $n^{p} < n$ for all $n \ge 2$, so $n^{-p} > n^{-1}$.
      Thus, the harmonic series provides a lower bound for $\sum n^{-p}$, and by
      comparison, $\sum n^{-p}$ diverges.
    \end{forwardimplication}

    \vspace{\baselineskip}
  
    \begin{backwardimplication}
      Assume now that $p > 1$. It is sufficient to  demonstrate that $\sum 2^{n}(2^{n})^{-p}$
      converges. Note that the partial sums $t_{k} = \sum_{n=1}^{k} 2^{n}(2^{n})^{-p}$
      form the partial sums of a geometric series with common ratio $r = 2^{1-p} < 1$:
      \begin{align*}
        t_{k} &= 1 + \frac{2}{2^{p}} + \frac{4}{4^{p}} + \cdots + \frac{2^{k}}{(2^{k})^{p}}\\
              &= 1 + \frac{1}{2^{p-1}} + \frac{1}{4^{p-1}} + \cdots + \frac{1}{(2^{k})^{p-1}}.
      \end{align*}
      Thus, $(t_{k})$ converges, and so too does $\sum n^{-p}$ by \lemref{lem:2.4.6}.
    \end{backwardimplication}
  \end{proof}
\end{problem}
