\section{Miscellaneous Problems}

\begin{problem}
  Let $P_{n} = \{ 1, 2, \dots, n \}$, and for any nonempty subset $S \subseteq P_{n}$, define
  \[
    v(S) = \dfrac{1}{|S|} \sum_{s \in S} s. 
  \]
  Also, define 
  \[
    \Scal_{n} = \{ v(A) - v(B) : A \cup B = P_n,\ A \cap B = \emptyset,\ A, B \neq \emptyset \},
  \]
  and 
  \[
    \diam{\Scal_{n}} = \sup \Scal_{n} - \inf \Scal_{n}.
  \]
  Show that $\lim\limits_{n \to \infty} \frac{\diam{\Scal_{n}}}{n} = 1$.

  \begin{callout}
    This is a slightly modified version of the problem posted by the user \texttt{chris\_34203}
    on the \href{https://discord.com/channels/1381996954237800608/1381996954673745932/1393174128432906357}{Aleph 0 Math Membership} Discord server.
  \end{callout}

\end{problem}

\begin{proof}
  We will show $\diam{\Scal_{n}} = n$ for all $n \ge 2$, which immediately implies the limit.

  First, notice the symmetry inherent in $\diam{\Scal_{n}}$: swapping $A$ and $B$ in any partition negates $v(A)-v(B)$, so
  \[
    \sup \Scal_{n} = -\inf \Scal_{n} \implies \diam{\Scal_{n}} = 2\sup \Scal_{n}.
  \]

  Next, we will bound $\sup \Scal_{n}$. Taking $A = \{ n \}$ and $B = P_n \setminus \{ n \}$ gives
  \[
    v(A) = n, \quad v(B) = \frac{1 + 2 + \cdots + (n-1)}{n - 1} = \frac{n}{2},
  \]
  which provides a lower bound for $\sup \Scal_{n}$ since $A, B$ were fixed:
  \[
    v(A) - v(B) = \frac{n}{2} \implies \sup \Scal_{n} \ge \frac{n}{2}.
  \]
  To show $\sup \Scal_{n} \le \frac{n}{2}$, choose any disjoint partition $A \cup B = P_{n}$ with $|A| = k$, so $1 \le k \le n - 1$.
  Let $S_A = \sum\limits_{a \in A} a$ and $S_B = \sum\limits_{b \in B} b$. Then,
  \[
    S_{A} \le n + (n-1) + \cdots + (n - k + 1) = \frac{k(2n - k + 1)}{2},
  \]
  and
  \[
    S_{B} \ge 1 + 2 + \cdots + (n - k) = \frac{(n - k)(n - k + 1)}{2}.
  \]
  Therefore, we have the desired upper bound for $\sup \Scal_{n}$ since $A, B$ were arbitrary:
  \[
    v(A) - v(B) = \frac{S_A}{k} - \frac{S_B}{n - k} \le \frac{2n - k + 1}{2} - \frac{n - k + 1}{2} = \frac{n}{2} \implies \sup \Scal_{n} \le \frac{n}{2}.
  \]

  Thus, we have shown $\frac{n}{2} \leq \sup \Scal_{n} \leq \frac{n}{2} \implies \diam{\Scal_{n}} = n$, and consequently,
  the limit follows.
\end{proof}

\begin{problem}
  
  If $\lim\limits_{n \to \infty} a_n = L > 0$, then show that $\lim\limits_{n \to \infty} \sqrt{a_n} = \sqrt{L}$. Be sure to discuss
  the issue when $\sqrt{a_{n}}$ makes sense.

  \begin{proof}
    Since $a_{n} \to L > 0$, there exists $N' \in \N$ such that
    $a_{n} \ge 0$ for all $n > N'$.  Hence every $\sqrt{a_{n}}$ that appears
    below is real, and $\sqrt{L}$ is real because $L > 0$.

    Let $\eps > 0$ be given. By convergence of $a_{n}$, choose $N \ge N'$ such that
    for all $n > N$,
    \[
      \abs{a_{n} - L}< \eps^{2}.
    \]
    Notice that 
    \begin{align*}
      \abs{a_{n} - L} &= \abs{(\sqrt{a_n} - \sqrt{L})(\sqrt{a_n} + \sqrt{L})} \\
                      &= \abs{\sqrt{a_n} - \sqrt{L}}\abs{\sqrt{a_n} + \sqrt{L}} \quad < \eps^{2},
    \end{align*}
    and that
    \[
      \abs{\sqrt{a_{n}} + \sqrt{L}} \geq \abs{\sqrt{a_n} - \sqrt{L}}.
    \]
    Taken together
    \[
      \abs{\sqrt{a_{n}} - \sqrt{L}}^{2} \leq \abs{\sqrt{a_{n}} - \sqrt{L}}\abs{\sqrt{a_{n}} + \sqrt{L}} < \eps^{2},
    \]
    which implies
    \[
      \abs{\sqrt{a_{n}} - \sqrt{L}} < \eps.
    \]

    Because $\eps > 0$ was arbitrary, we conclude
    \[
      \lim_{n \to \infty} \sqrt{a_{n}} = \sqrt{L}.
    \]
  \end{proof}

\end{problem}

\begin{problem}
  Suppose functions $f : \R \to \R$ and $g : \R \to \R$ are continuous and $f(p) = g(p)$ for all all $p \in \Q$. Prove that
  $f = g$.

  \begin{callout}
    This problem comes from the \texttt{Fee Phi Fo Fum} \href{https://www.youtube.com/watch?v=OuwLcZQsDoY}{YouTube channel}.
  \end{callout}

  \begin{proof}
    We already know $f = g$ over the rationals, so it is sufficient to show $f = g$ for all irrational numbers.

    Let $x \in \I$ be arbitrary. Consider the sequence $(a_{n})_{n=1}^{\infty}$. Because
    $\Q$ is dense in $\R$ (by the result of \probref{prob:infinite-rationals-in-reals}), we have
    \[
      \forall n \in \N.\, \exists q_{n} \in \Q.\, x < q_{n} < x + a_{n},
    \]
    and $(q_{n}) \to x$ as $n \to \infty$. By continuity, we obtain
    \[
      f(x) = \lim_{n \to \infty} f(q_{n}) = \lim_{n \to \infty} g(q_{n}) = g(x).
    \]
    Thus, $f(x) = g(x)$ for all $x \in \I$ (since $x \in \I$ was arbitrary), and $f = g$ for all real numbers.
  \end{proof}
\end{problem}

