\section{Limits of Functions}

\begin{problem}
  Prove the following limits:

  \begin{enumerate}[label=(\alph*)]
    \item $\lim\limits_{x \to 2} 2x + 4 = 8$ 
      
      \begin{proof}
        Let $f(x) = 2x + 4$ and $\eps > 0$ be given. Let $\delta = \frac{\eps}{2}$. Then, it follows that: 
        \[
          \abs{x - 2} < \frac{\eps}{2} \implies \abs{f(x) - 8} = \abs{2x + 4 - 8} = \abs{2x - 4} = 2\abs{x - 2} < 2\cdot\frac{\eps}{2} = \eps.
        \]
        Thus, we have shown that $f(x) \to 8$ as $x \to 2$. 
      \end{proof}

    \item $\lim\limits_{x \to 0} x^{3} = 0$ 
      \begin{proof}
        Let $f(x) = x^{3}$ and $\eps > 0$ be given. Let $\delta = \sqrt[3]{\eps}$. Then, it follows that: 
        \[
          \abs{x - 0} < \sqrt[3]{\eps} \implies \abs{f(x) - 0} = \abs{x^{3}} < (\sqrt[3]{\eps})^{3} = \eps.
        \]
        Thus, we have shown that $f(x) \to 0$ as $x \to 0$. 
      \end{proof} 

    \item $\lim\limits_{x \to 2} x^{3} = 8$ 
      \begin{proof}
        Let $f(x) = x^{3}$ and $\eps > 0$ be given. We can show that $\delta = \min{(1, \frac{\eps}{19})}$
        satisfies the definition of the limit. Consider the two cases:
        \begin{itemize} 
          \item When $\eps \leq 19$, then $\delta = \frac{\eps}{19} \leq 1$. For all $x \in \R$ 
            such that $0 < \abs{x - 2} < \frac{\eps}{19}$, we can show that
              \begin{align*}
                \abs{f(x) - 8} &= \\
                \abs{(x-2)(x^{2} +2x + 4)} &= \\
                \abs{x-2} \abs{x^{2} + 2x + 4} &< \frac{\eps}{19} \abs{x^{2} + 2x + 4} \\ 
                                               &< \eps.
              \end{align*}

              The last inequality holds because $\abs{x^{2} + 2x + 4} < 19$ for all
              $x$ such that $0 < \abs{x - 2} < \delta \leq 1$.
              (We leverage this fact for the second case as well.)

            \item When $\eps > 19$, then $\delta = 1$. Then, for all $x$
              such that $0 < \abs{x - 2} < 1$, it follows that
              \begin{align*}
                \abs{f(x) - 8} &= \\
                \abs{(x-2)(x^{2} +2x + 4)} &= \\
                \abs{x-2} \abs{x^{2} + 2x + 4} &< \abs{x^{2} + 2x + 4} \\ 
                                               &\leq 19 \\ 
                                               &< \eps.
              \end{align*}
        \end{itemize}
        Thus, we have shown that $f(x) \to 8$ as $x \to 2$. 
      \end{proof}

    \item $\lim\limits_{x \to \pi} \lfloor x \rfloor = 3$ 
      \begin{proof}
        Let $f(x) = \lfloor x \rfloor$ and $\eps > 0$ be given. We then fix $\delta = 0.1$ to show
        that $\abs{f(x) - 3} < \eps$. For all $x \in (\pi - 0.1, \pi) \cup (\pi, \pi + 0.1)$, we immediately observe
        \[
          \abs{f(x) - 3} = \abs{3 - 3} = 0 < \eps.
        \]
        Thus, we have shown that $f(x) \to 3$ as $x \to \pi$.
      \end{proof}

  \end{enumerate}
\end{problem}


\begin{problem}
  \label{prob:smaller-delta-sufficient}
  Prove the following statement:
    \begin{displayquote}
      Assume a particular $\delta > 0$ has been constructed as a suitable response to
      a particular $\eps > 0$ challenge. Then, any smaller $\delta$ will also suffice.
    \end{displayquote}

  \begin{proof}
    Given an $\eps > 0$ and some $\delta > 0$ satisfying the definition of a
    limit (say, for a function $f(x)$ whereby the limit $L \in \R$), we can let
    $0 < \delta' < \delta$. Then, we know, for all $x \in (c - \delta, c) \cup (c, c + \delta)$, that
    $\abs{f(x) - L} < \eps$. Since $\delta' < \delta$, we also have $\abs{f(x) - L} < \eps$ whenever
    $x \in (c - \delta', c) \cup (c, c + \delta')$ because 
    $(c - \delta', c) \cup (c, c + \delta') \subset (c - \delta, c) \cup (c, c + \delta)$.
  \end{proof}
\end{problem}


\begin{problem}
  The statement $\lim\limits_{x \to 0} \dfrac{1}{x^{2}} = \infty$ certainly makes
  intuitive sense. Construct a rigorous definition in the "challenge-response"
  style for a limit statement of the form
    \[
      \lim_{x \to c} f(x) = \infty
    \]
  and use it to prove the previous statement.

  \begin{definition}[Vertical Limit]
    \label{def:limit-to-infty-x-to-finite-real}
    Let function $f : S \to \R$ where $S \subseteq \R$ and, for some $c \in \R$, there exists a
    sequence $(x_{n}) \subseteq S \setminus \{ c \}$ such that $(x_{n}) \to c$ as $n \to \infty$.
    Then, we say $\lim\limits_{x \to c} f(x) = \infty$ if and only if for any
    $M > 0$, there exists a $\delta > 0$ such that
    \[
      \forall x \in S.\, 0 < \abs{x - c} < \delta \implies f(x) > M.
    \]
  \end{definition}

  \begin{proof}
    Let $f(x) = \dfrac{1}{x^{2}}$. (Certainly, it is clear that
    $(x_{n})_{n}^{\infty} = \frac{1}{n}$ is contained by $S \setminus \{ 0 \}$
    and that $(x_{n}) \to 0$ as $n \to \infty$, so the domain conditions for 
    \defref{def:limit-to-infty-x-to-finite-real} are met.)

    Now let $M > 0$ be given and fix $\delta = \dfrac{1}{\sqrt{M}}$. 
    Then, for all $x \in S$,
    \[
      0 < \abs{x} < \delta \implies x^{2} < \frac{1}{M} \implies  f(x) = \frac{1}{x^{2}} > M.
    \]
  \end{proof}

\end{problem}

\begin{problem}
  Construct a definition for the statement $\lim\limits_{x \to \infty} f(x) = L$.
  Show $\lim\limits_{x \to \infty} \dfrac{1}{x} = 0$.

  \begin{definition}[Horizontal Limit]
    \label{def:limit-to-finite-real-x-to-infinite}
    Let function $f : S \to \R$ where $S \subseteq \R$ and, for any $N > 0$, there 
    exists some $x \in S$.
    Then, we say $\lim\limits_{x \to \infty} f(x) = L$ if and only if for any
    $\eps > 0$, there exists some $N > 0$ such that
    \[
      \forall x \in S.\, x > N \implies \abs{f(x) - L} < \eps.
    \]
  \end{definition}

  \begin{proof}
    Let $f(x) = \dfrac{1}{x}$. (Certainly, it is clear that $S$ is unbounded
    above, so the domain conditions for
    \defref{def:limit-to-finite-real-x-to-infinite} are met.)

    Now let $\eps > 0$ be given and fix $N = \dfrac{1}{\eps}$. 
    Then, for all $x \in S$,
    \[
      x > N = \frac{1}{\eps} \implies f(x) = \frac{1}{x} < \eps,
    \]
    so we have shown $\lim\limits_{x \to \infty} f(x) = 0$.
  \end{proof}

\end{problem}

\begin{problem}
  What would a rigorous definition for 
  $\lim\limits_{x \to \infty} f(x) = \infty$ look like? 
  Give an example of such a limit.

  \begin{definition}[Horizontal Divergence]
    \label{def:limit-to-infty-x-to-infinite}
    Let function $f : S \to \R$ where $S \subseteq \R$ and, for any $N > 0$, there 
    exists some $x \in S$.
    Then, we say $\lim\limits_{x \to \infty} f(x) = \infty$ if and only if for any
    $M > 0$, there exists some $N > 0$ such that
    \[
      \forall x \in S.\, x > N \implies f(x) > M.
    \]
  \end{definition}

  Consider $f(x) = x$. It can be shown that $\lim\limits_{x \to \infty} f(x) = \infty$.

\end{problem}

\begin{problem}
  \label{prob:squeeze-theorem}
  Let $f(x) \leq g(x) \leq h(x)$ for all $x$ in some common domain $A$. Suppose $\lim\limits_{x \to c} f(x) = L$ and
  $\lim\limits_{x \to c} h(x) = L$ at some $c \in A$. Use the $\eps$-$\delta$  definition of the limit
  to show  that $\lim\limits_{x \to c} g(x) = L$ as well.

  \begin{proof}
    Let $\eps > 0$ be given. We need to show that there exists some $\delta > 0$ such that
    \[
      \forall x \in A.\, 0 < \abs{x - c} < \delta \implies \abs{g(x) - L} < \eps.
    \]

    By the hypothesis and the result of \probref{prob:smaller-delta-sufficient}, we know there exists
    $\delta_{1}$ and $\delta_{2}$ such that, for all $x \in A$ where $0 < \abs{x - c} < \min{(\delta_{1}, \delta_{2})}$,
    $\abs{f(x) - L} < \eps$ and $\abs{h(x) - L} < \eps$, respectively. Set $\delta = \min{(\delta_{1}, \delta_{2})}$. 
    Then, we have
    \[
      L - \eps < f(x) \leq g(x) \leq h(x) < L + \eps \implies \abs{g(x) - L} < \eps,
    \]
    for all $x \in A$ where $0 < \abs{x - c} < \delta$, so, by definition, we have 
    shown $\lim\limits_{x \to c} g(x) = L$.
  \end{proof}
  
\end{problem}
